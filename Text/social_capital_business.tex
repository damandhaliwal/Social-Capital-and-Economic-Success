\documentclass[11pt]{article}
 
\usepackage[margin=1in]{geometry} 
\usepackage{amsmath,amsthm,amssymb,bm}
\usepackage{ dsfont }
\usepackage{ bbold }
\usepackage{enumerate}
\newcommand{\N}{\mathbb{N}}
\newcommand{\Z}{\mathbb{Z}}
\usepackage{setspace}
\usepackage{titlesec}
\usepackage{hyperref}
\usepackage{booktabs}
\usepackage{adjustbox}
\usepackage{graphicx}
\usepackage{float}
\usepackage{natbib}
\usepackage{threeparttable}
\usepackage{enumitem}
\usepackage[T1]{fontenc}

\setstretch{1.2}

\begin{document}
% Title page
\begin{titlepage}
    \centering
    \vspace*{2cm}
    
    \vspace{0.5cm}

    \Large{\textbf{The Role of Social Capital in Business Performance:}}\\  
    \Large{An Empirical Analysis}  
    \vspace{0.5cm}

    \Large{\textbf{Author Name:}}\\
    \Large{Damanveer Singh Dhaliwal}\\
    \Large{Student ID: 1012787147}
    
    \vfill
    
    \large{Department of Economics}\\
    \large{University of Toronto}
    
    \vspace{0.8cm}
    
    \large{\today}
\end{titlepage}

\section{Executive Summary}

\pagebreak
\section{Introduction}
Social Capital, the strength of an individual's social networks and the resources accessible through them, has been the subject of extensive research in sociology and economics. Economic intuition and conventional research suggests that social capital is a productive asset that reduces transaction costs and smooths friction in economic exchanges. [CITE] However, this concept has been difficult to quantify empirically due to the lack of comprehensive data on social networks at scale.

Until recently, the primary data for studying social capital was the National Longitudinal Study of Adolescent to Adult Health (Add Health) dataset which covered around 20,000 participants in 132 schools in the United States. The limited data made it difficult for researchers to study how social capital plays a central role in shaping social phenomena like income equality and economic opportunity.

Recently, \cite{chetty_social_2022-1} utilized a novel dataset from Facebook that included over 21 billion social connections to estimate social capital at the local level across the United States. This data allowed the authors to construct a social capital index and examine its relationship with the economic mobility of individuals across different regions. Their findings indicated a strong positive correlation between social capital and economic mobility, suggesting that individuals in areas with higher social capital tend to have better economic outcomes.

This paper aims to build upon the findings of \cite{chetty_social_2022-1} by exploring the relationship between social capital and business performance, which has not been extensively studied in the existing literature. We will investigate two aspects of business performance: (1) the likelihood of business survival given an exogenous shock (the COVID-19 pandemic), and (2) the revenue growth of businesses over time. By analyzing these dimensions, we aim to provide a comprehensive understanding of how social capital influences business outcomes.

The research question is motivated by two main questions: First, does social captial contribute to the prosperity of counties by enhancing the performance of local businesses? Second, can social capital act as a buffer for businesses during economic downturns or crises, such as the COVID-19 pandemic? The results of this study could have significant implications for policymakers and business leaders, as they highlight the importance of fostering social capital within communities to support economic growth and resilience as well as inform strategies for capital allocation. The study may also motivate individals and businesses to invest in building and maintaining strong social networks.

The paper is structures as follows: Section 2 reviews related literature in depth and defines the scope of the study in the context of existing research. Section 3 describes the data sources, including the social capital index, in depth and outlines the empirical model and identification strategy. Section 4 presents the main results and robustness checks. Finally, Section 5 concludes with a summary of findings and discusses potential avenues for future research.

\section{Literature Review}

\section{Empirical Model and Data}

\section{Results}

\section{Conclusion}

\pagebreak
\bibliography{references}
\bibliographystyle{chicago}
\end{document}