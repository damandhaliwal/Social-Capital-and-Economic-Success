\documentclass[11pt]{article}
 
\usepackage[margin=1in]{geometry} 
\usepackage{amsmath,amsthm,amssymb,bm}
\usepackage{ dsfont }
\usepackage{ bbold }
\usepackage{enumerate}
\newcommand{\N}{\mathbb{N}}
\newcommand{\Z}{\mathbb{Z}}
\usepackage{setspace}
\usepackage{titlesec}
\usepackage{hyperref}
\usepackage{booktabs}
\usepackage{adjustbox}
\usepackage{graphicx}
\usepackage{float}
\usepackage{natbib}
\usepackage{threeparttable}
\usepackage{enumitem}
\usepackage[T1]{fontenc}

\setstretch{1.5}

\begin{document}
% Title page
\begin{titlepage}
    \centering
    \vspace*{2cm}
    
    \vspace{0.5cm}

    \Large{\textbf{The Role of Social Capital in Business Performance}}\\  
    \Large{An Empirical Analysis}  
    \vspace{0.5cm}

    \Large{\textbf{Author Name:}}\\
    \Large{Damanveer Singh Dhaliwal}\\
    %\Large{Student ID: 1012787147}
    
    \vfill
    
    \large{Department of Economics}\\
    \large{University of Toronto}
    
    \vspace{0.8cm}
    
    \large{\today}
\end{titlepage}

\section{Introduction}
Social Capital, the strength of an individual's social networks and the resources accessible through them, has been the subject of extensive research in sociology and economics. Economic intuition and conventional research suggests that social capital is a productive asset that reduces transaction costs and smooths friction in economic exchanges. [CITE] However, this concept has been difficult to quantify empirically due to the lack of comprehensive data on social networks at scale.

Until recently, the primary data for studying social capital was the National Longitudinal Study of Adolescent to Adult Health (Add Health) dataset which covered around 20,000 participants in 132 schools in the United States. The limited data made it difficult for researchers to study how social capital plays a central role in shaping social phenomena like income equality and economic opportunity.

Recently, \cite{chetty_social_2022-1} utilized a novel dataset from Facebook that included over 21 billion social connections to estimate social capital at the local level across the United States. This data allowed the authors to construct a social capital index and examine its relationship with the economic mobility of individuals across different regions. Their findings indicated a strong positive correlation between social capital and economic mobility, suggesting that individuals in areas with higher social capital tend to have better economic outcomes.

This paper aims to build upon the findings of \cite{chetty_social_2022-1} by exploring the relationship between social capital and business performance, which has not been extensively studied in the existing literature. We will investigate two aspects of business performance: (1) the likelihood of business survival given an exogenous shock (the COVID-19 pandemic), and (2) the revenue growth of businesses over time. By analyzing these dimensions, we aim to provide a comprehensive understanding of how social capital influences business outcomes.

The research question is motivated by two main questions: First, does social captial contribute to the prosperity of counties by enhancing the performance of local businesses? Second, can social capital act as a buffer for businesses during economic downturns or crises, such as the COVID-19 pandemic? The results of this study could have significant implications for policymakers and business leaders, as they highlight the importance of fostering social capital within communities to support economic growth and resilience as well as inform strategies for capital allocation. The study may also motivate individals and businesses to invest in building and maintaining strong social networks.

The paper is structured as follows: Section 2 reviews related literature in depth and defines the scope of the study in the context of existing research. Section 3 describes the data sources, including the social capital index, in depth and outlines the empirical model and identification strategy. Section 4 presents the main results and robustness checks. Finally, Section 5 concludes with a summary of findings and discusses potential avenues for future research.

\section{Literature Review}
Social capital has been a topic of interest in various fields, including sociology, economics, and political science. The origins of social capital theory can be traced back to \cite{bourdieu2021forms} who defined social capital as the aggregate of actual or potential resources linked to a durable network of institutionalized relationships. Importantly, the author posits that social capital requires extensive investment of time and energy to create and maintain. This signals that social capital might not be evenly distributed across individuals or communities, leading to disparities in access to resources and opportunities. It also suggests that social capital might not be easily transferable or liquidated, however, it is unsure if social capital is transferred intergenerationally.

The idea was then studied extensively by other scholars such as \cite{putnam_social_1994} who drew on a multi-decade study of regional governments in Italy to argue that regions with deep traditions of civic engagement possess a civic community that serves as a foundation for economic development and democratic governance. Putnam emphasized that social capital increases with use, in contrast to physical capital which depreciates with use. He further argues, based on survey data, that erosion of trust in the United States has led to urban decay, decline in education and loss of trust in the government. These claims are not rooted in strong empirical evidence and have necessitated further research.

Over the years, there have been numerous economic studies aiming to model and quantify social capital and its effects on various economic outcomes. \cite{dasgupta2003social} provides a rigorous economic model of social capital. The central contribution of the paper is to show how social capital can be modeled into standard macroeconomic production functions. The author also argues that dense social networks can inhibit the flow of capital and labour and might lead to equilibria that are suboptimal for economic growth. 

More recently, \cite{chetty_social_2022-1} and \cite{chetty_social_2022-2} utilized a novel dataset from Facebook that included over 21 billion social connections and modern computational techniques to form a first of its kind social capital index at the college, county, high school, and zip code levels across the United States. The authors meaningful results that revitalized interest in social capital research. In particular, they found that children who grow up in areas where cross-class interactions are common tend to have signficantly higher upward income mobility. They also found that social capital tends to be highly stratified by socioeconomic status and that differences in this economic connectedness accounts for the relationships between upward mobility and factors such as poverty and racial segreagation.

There has also been research exploring the relationship between social capital and public health outcomes. \cite{makridis_how_2021} investigate whether areas with higher social capital experienced better health outcomes during the COVID-19 pandemic. They found that moving from the 25th to the 75th percentile of social capital is associated with a 18\% reduction in COVID-19 cases and a 5.7\% reduction in COVID-19 deaths. 

On similar lines exploring social capital's role during crisis, \cite{lins_social_2017} examine how social capital that is generated by Corporate Social Responsibility (CSR) initiatives can help firms weather economic downturns. Using data from the 2008-2009 financial crisis, they find that firms with high social capital, as measured by their CSR activities, experienced significantly better stock performance during the crisis compared to firms with low social capital. The authors argue that social capital built through CSR initiatives enhances trust and cooperation among stakeholders, which can be particularly valuable during times of economic uncertainty.

This paper aims to overcome some of the limitations of previous research and contribute to the literature by exploring the relationship between social capital and business performance. While previous studies have primarily focused on individual-level outcomes or public health, this study will investigate how social capital influences business survival in times of a crisis (the COVID-19 pandemic) and also how it affects revenue growth over time. We will rely on the social capital index developed by \cite{chetty_social_2022-1} to conduct our analysis, thereby building on their foundational work.

\section{Empirical Model and Data}
\subsection{Data}
There are two primary data sources for this study:
\subsubsection{Social Capital Atlas}
The Social Capital Atlas developed by \cite{chetty_social_2022-2} provides a comprehensive measure of social capital at the college, county, high school and zip code level. The index is constructed using data from Facebook users with the following attributes: US residents aged between 25-44; active on Facebook at least once in the 30 days preceding data collection; have at least 100 friends on Facebook; non-missing ZIP Code. This paper relies only on the county level data so we will not discuss the other levels in detail. 

The social capital index consists of three main components: (1) Economic Connectedness (EC), which measures the extent of cross-class interactions; (2) Social Cohesion, which captures the density and clustering of social networks within a community; and (3) Civic Engagement, which reflects participation in civic activities such as volunteering and voting. 

Economic Connectedness was calculated by first defining the socioeconomic status (SES) of individuals based on different variables such as income, wealth, educational attainment, occupation, family background, neighborhood and consumption. These measures were combined into a SES index using a machine learning algorithm and subsequently each individual was assigned a SES percentile rank within their birth cohort. Economic Connectedness was then measured as the share of friends an individual has from the other half of the SES distribution.

Social Cohesion included measures such as clustering, which indicate the extent to which friends of friends are also friends, and support ratio, which measures the rate at which pairs of friends in a community have other friends in common. Civic Engagement was based on a proxy variable that measured the rate of volunteering in each area using membership in Facebook groups related to volunteering activities.

\subsubsection{United States Historical Business Data}
To complement the social capital data, we utilize the United States historical business data from \cite{SP3/GY5K1C_2022}. This dataset contains detailed information on business establishments across the United States, including data on industry classification, employment, revenue, and survival status over time. The data spans multiple years, allowing for longitudinal analysis of business performance and is collected by Data Axle, a leading provider of business data and analytics.

The two datasets are merged using county FIPS codes to create a comprehensive dataset that includes both social capital measures and business performance metrics at the county level. The social capital index was constructed in 2022, which limits the temporal scope of the analysis. It has been assumed that social capital exhibited relative stability over the short to medium term, allowing for its application in analyzing business performance during the COVID-19 pandemic period (2020-2021). The Data Axle data was used for the years 2019-2024 to capture business performance before, during, and after the pandemic.

\subsection{Identification Strategy}
We are interested in two main outcomes related to business performance: (1) business survival during the COVID-19 pandemic, and (2) revenue growth over time. 

We rely on the COVID-19 pandemic as a natural experiment to identify the causal impact of social capital on business resilience. The pandemic represents an exogenous shock that affect all US counties simultaneously, while they varied in their pre-existing levels of social capital. We hypothesize that if social capital provides economic resilience, high social capital counties should experience smaller declines in firm survival and higher revenue growth. 

Our identification strategy involves a cross-sectional analysis of business performance with heterogeneous treatment intensity. The key identifying assumption is parallel trends: without the pandemic, firms in high- and low- social capital counties would have followed similar survival trajectories. We provide suggestive evidence for this assumption using pre-period data (2016-2019). 

\subsection{Key Variables}
The primary outcome variables of interest are firm survival, a binary indicator equal to 1 if a firm present in 2019 remains in the Data Axle database in 2024, and sales growth, the log change in sales from 2019 to 2024, conditional on survival. We chose to study survival over the next five years to capture any lagged effects of the pandemic on business performance since some businesses may have initially survived but failed later due to prolonged economic challenges.

We examine all three components of the social capital index: Economic Connectedness (EC), Social Cohesion, and Civic Engagement. While \cite{chetty_social_2022-1} found EC to be the strongest predictor of individual mobility, it is an empirical question whether the same holds for firm-level outcomes.

The control variables include log employees in 2019 (for survival specifications) or log sales in 2019 (for growth specifications), along with state fixed effects and two-digit NAICS industry fixed effects. We include only one size control to avoid multicollinearity between sales and employment

\subsection{OLS Specification}
We begin with a simple OLS specification to establish the average treatment effects. For the survival, we estimte:
\begin{equation}
    Survival_{i,2024} = \alpha + \beta \cdot SC_{c} + \gamma \cdot ln(Emp_{i,2019})  + \theta_s + \mu_j + \epsilon_{ic}
\end{equation}
where $Survival_{i,2024}$ is a binary indicator for whether firm $i$ in county $c$ survived until 2024, $SC \in \{EC_c, COH_c, CIV_c\}$ and $EC_c$, $COH_c$, and $CIV_c$ are the Economic Connectedness, Social Cohesion, and Civic Engagement components of the social capital index for county $c$, respectively. $ln(Emp_{i,2019})$ is the log of employees for firm $i$ in 2019, $\theta_s$ are state fixed effects, $\mu_j$ are two-digit NAICS industry fixed effects, and $\epsilon_{ic}$ is the error term. Here $i$ indexes firm, $c$ indexes county, $s$ indexes state, and $j$ indexes industry. Standard errors are clustered at the county level to account for potential correlation of errors within counties.

To assess which dimensions of social capital are most relevant for business survival, we also estimate a specification that includes all three components of the social capital index simultaneously:  
\begin{equation}
    Survival_{i,2024} = \alpha + \beta_1 \cdot EC_{c} + \beta_2 \cdot COH_c + \beta_3 \cdot CIV_c + \gamma \cdot ln(Sales_{i,2019})  + \theta_s + \mu_j + \epsilon_{ic}
\end{equation}

For revenue growth, we estimate the following specification:
\begin{equation}
    \Delta ln(Sales_{i,2019-2024}) = \alpha + \beta_1 \cdot EC_{c} + \beta_2 \cdot COH_c + \beta_3 \cdot CIV_c + \gamma \cdot ln(Sales_{i,2019})  + \theta_s + \mu_j + \epsilon_{ic}
\end{equation}
where $\Delta ln(Sales_{i,2019-2024})$ is the change in log sales from 2019 to 2024 for firm $i$ in county $c$. The other variables are defined as above.

We note that the growth specification conditions on firm survival. To the extent that social capital affects selection into survival, these coefficients capture effects among a non-random subsample of surviving firms

\subsection{Double/Debiased Machine Learning (DML)}
OLS with high-dimensional controls risks omitted variable bias if we exclude relevant confounders and regularization bias. Double/Debiased Machine Learning (DML) (\cite{chernozhukov_doubledebiased_2024}) addresses both concerns by using machine learning to flexibly partial out confounders while preserving valid inference on treatment effects.

We use DML for two purposes: 
    \vspace{-0.3cm}
\begin{enumerate}
    \item to estimate the causal effect of social capital dimension with flexible controls, and
    \vspace{-0.3cm}
    \item to explore heterogeneous treatment effects by industry.
\end{enumerate}

For our industry heterogeneity analysis, we focus on social cohesion as the treatment variable, with economic connectedness and civic engagement included as controls in $X_i$. This choice reflects our baseline finding that cohesion is the dominant predictor of firm survival.

We implment the Partially Linear Regression (PLR) DML estimator for social cohesion as follows:
\begin{equation}
    Y_i = \theta \cdot COH_c + g(X_i) + \epsilon_i
\end{equation}
\begin{equation}
    SC_c = m(X_i) + \nu_i
\end{equation}
where $Y_i$ is the firm survival, $COH_i$ is the social cohesion measure, $X_i$ is a vector of control variables (log employees in 2019, state-fixed effects, two-digit NAICS industry fixed effects), and $g(\cdot)$ and $m(\cdot)$ are unknown functions estimated using machine learning methods. 

We use XGBoost (Gradient Boosted Trees) to estimate $g(\cdot)$ and $m(\cdot)$ due to its strong performance in practice. 
\begin{itemize}
    \item $\hat{g}(X_i) = \mathbb{E}\!\left[Y_i \mid X_i\right]$ is the outcome model
    \item $\hat{m}(X_i) = \mathbb{E}\!\left[SC_i \mid X_i\right]$ is the treatment model
\end{itemize}

Then we compute the residuals:
\begin{equation}
    \tilde{Y}_i = Y_i - \hat{g}(X_i)
\end{equation}
\begin{equation}    \tilde{COH}_i = COH_i - \hat{m}(X_i)
\end{equation}

Finally, we estimate the treatment effect $\theta_{SC}$ by regressing the residualized outcome on the residualized treatment:
\begin{equation}
    \tilde{Y}_i = \theta_{SC} \cdot \tilde{COH}_i + \tilde{\epsilon}_i
\end{equation}
To avoid overfitting bias, we use 5-fold cross-fitting. For each fold, nuisance functions are estimated on the complement fold and used to compute residuals on the held-out fold. This ensures all predictions are out-of-sample.

A central contribution of this paper is to document the heterogeneity in treatment effects across industries. To do so, we extend the DML framework to by estimating separate DML models for each two-digit NAICS industry and each 4-digit NAICS sub-industry. These are specified as follows:
\begin{equation}
    Y_{ij} = \theta_{j} \cdot COH_c + g_j(X_{ij}) + \epsilon_{ij}
\end{equation}
where $j$ indexes industry or sub-industry. This allows us to estimate industry-specific treatment effects $\theta_{j}$ while flexibly controlling for confounders within each industry.

The sub-industry level analysis is novel and is conducted to uncover whether granular findings are being masked at the broader industry or county level. This is particularly important for policy implications, as it can inform targeted interventions to foster social capital in specific sectors where it has the most significant impact on business performance.

\subsection{Quantile Regression}
Finally, to explore how the effect of social capital varies across the distribution of firm performance, we implement quantile regression. This approach allows us to estimate the impact of social capital on different points of the outcome distribution, such as the median or the 25th and 75th percentiles. For firm performance, we estimate the following quantile regression model:
\begin{equation}
    Q_{\tau}(Y_i \mid SC_c, X_i) = \alpha^{\tau} + \beta_1^{\tau} \cdot EC_{c} + \beta_2^{\tau} \cdot COH_c + \beta_3^{\tau} \cdot CIV_c + \gamma^{\tau} \cdot ln(Sales_{i,2019})
\end{equation}
where $Q_{\tau}(Y_i \mid SC_c, X_i)$ denotes the $\tau$-th conditional quantile of the outcome variable $Y_i$ given social capital measures and control variables.

Quantile regression allows us to test competing hypotheses about social capital's distributional effects. If social capital provides downside protection, we expect larger effects at lower quantiles. If social capital amplifies success, we expect larger effects at upper quantiles.

\section{Empirical Results}


\section{Conclusion}

\pagebreak
\bibliography{references}
\bibliographystyle{chicago}
\end{document}